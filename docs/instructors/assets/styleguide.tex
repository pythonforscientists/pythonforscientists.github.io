\documentclass{article}
\usepackage[utf8]{inputenc}
\usepackage{listings}
\usepackage{xcolor}
\definecolor{commentgreen}{RGB}{2,112,10}
\definecolor{eminence}{RGB}{108,48,130}
\definecolor{weborange}{RGB}{255, 100, 23}
\definecolor{frenchplum}{RGB}{129,20,83}

\lstset{ % These are defaults
    basicstyle=\ttfamily, % set font style
    frame=tb, % draw a frame at the top and bottom of the code block
    tabsize=2, % tab space width
    showstringspaces=false, % don't mark spaces in strings
    numbers=left, % display line numbers on the left
}

\lstdefinestyle{pippython}{
    language=Python,
    commentstyle=\color{commentgreen}, % comment color
    keywordstyle=\color{eminence}, % keyword color
    stringstyle=\color{red}, % string color
    emphstyle=\color{blue}, % bool int, emphasis
    % keyword highlighting
    classoffset=1, % starting new class
    otherkeywords={>,<,.,;,-,!,=,~,/,*,+},
    morekeywords={>,<,.,;,-,!,=,~,/,*,+},
    keywordstyle=\color{weborange},
    breaklines=true, % line breaks for wrapping
}

\title{Style Guide - Python for Scientists}

\begin{document}

\maketitle

Use descriptive names, even if it increases line length slightly. count is more descriptive than c.

In general, avoid using single character variable names, since they are often difficult to follow and read. Never use the characters ‘l’ (lowercase letter el), ‘O’ (uppercase letter oh), ‘I’ (uppercase letter eye), ‘1’ (number one), or ‘0’ (number zero) as single character variable names. Avoid using ‘L’ (uppercase letter el) when possible. In some fonts, these characters are indistinguishable from the numerals one and zero.

\section*{Capitalization Conventions}
\begin{itemize}
    \item Variables and objects: camelCase (\verb|count|, \verb|numRuns|)
    \item Functions: snake\_case (\verb|sum|, \verb|sum_of|, \verb|get_result|)
    \item Classes: CapCase (\verb|GameScore|, \verb|Runs|)
    \item Constants: ALLCAPS (\verb|PI|, \verb|FIELDLENGTH|)
\end{itemize}

\section*{Whitespace}
Use whitespace wisely. Remember, whitespace takes the form of both horizontal whitespace (spaces and indentation) and vertical whitespace (blank lines). Both too much and too little whitespace make your source code difficult to read.

Leave one space around initializations and boolean operators.

\begin{lstlisting}[style=pippython]
runs = 1 # Good
if (runs >= 10): # Good
runs=3 # Bad
\end{lstlisting}

Observe how the equal sign in line 1 is surrounded by spaces. This is an example of space around initialization.

Also observe how the greater than/equal to sign in line 2 is surrounded by spaces without a space between components of the boolean operator. This ensures that the syntax is correct for the entire boolean operator (the \verb|>=| is one unit, not a separate \verb|>| and \verb|=|) while still providing adequate whitespace. This is an example of space around a boolean operator.

Also leave space before and after comment demarcations, as shown in lines 1-3. The comment demarcation in Python is a \verb|#|, and there is a space before and after.

Leave an extra space between function arguments. Do not leave an extra space before or after function parentheses.

\begin{lstlisting}[style=pippython]
atlRuns = GetRuns('ATL') # Good
ariWins = GetWins('ARI', 'away') # Good
 
bosRuns = GetRuns( 'BOS' ) # Bad, too much space around args
chiWins = GetWins( 'CHI', 'home' ) # Bad, too much space around args
dalWins = GetWins('DAL','away') # Bad, no space between args
\end{lstlisting}

\section*{Indentation}

In connection with whitespace, make sure you follow indentation conventions for your language. Python enforces indentation, so make sure you use consistent indentation.

Indent using one tab, which should indent two spaces.

Indent anything nested, including function contents, logic statement bodies, loops, and nested objects (mainly arrays, lists, and dictionaries).

Do not put a space before a colon in a conditional or logic statement.

\begin{lstlisting}[style=pippython]
if (numRuns > 3): # Good
  print("More than three runs!") # Good, two spaces of indentation
else : # Bad
   print("Not more than three runs.") # Bad, inconsistent indentation (3 spaces)
\end{lstlisting}

Soft-wrap lines in your editor, not by manually splitting a line into multiple lines. Not everyone's editor window size and font size is the same as yours. 

\end{document}
